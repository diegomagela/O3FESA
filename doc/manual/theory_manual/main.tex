\documentclass[12pt]{article}

% packages
\usepackage{amsmath}
\numberwithin{equation}{section}
\usepackage{hyperref}

% font
\usepackage{stix}
\usepackage[T1]{fontenc}

% new commands

% partial derivative
% #1 - numerator
% #2 - denominator
\newcommand{\pdiff}[2]{ \frac{\partial #1}{ \partial #2} }
\newcommand{\dpdiff}[2]{ \dfrac{\partial #1}{ \partial #2} }

% element notation
\def\el{{ \mathcal{E} }}

% increase line space in B(b)matrix environment
% \renewcommand{\arraystretch}{1.75}



\title{O3FESA}
\author{Diego Magela Lemos}
\date{version 0.1 \\ \today}

\begin{document}

\maketitle

\section{What is O3FESA?}

O3FESA (\textbf{O}pen-source \textbf{O}bject-\textbf{O}riented \textbf{F}inite \textbf{E}lement \textbf{S}hell \textbf{A}nalysis) is a software written in \texttt{C++} using the objected-oriented programming paradigm.

\section{Shell theories}

\subsection{First-order shear deformation theory}

\subsubsection{Displacement field}

\begin{subequations}
    \begin{align}
        u_1(x, y, z, t) & = u(x, y, t) + z \phi_x(x, y, t) \\
        u_2(x, y, z, t) & = v(x, y, t) + z \phi_y(x, y, t) \\
        u_3(x, y, z, t) & = w(x, y, t)
    \end{align}
    %
    \label{eq:displacement_field}
\end{subequations}
%
The displacement vector is defined as
%
\begin{equation}
    q =
    \begin{Bmatrix}
        u      \\
        v      \\
        w      \\
        \phi_x \\
        \phi_y
    \end{Bmatrix}
    %
    \label{eq:displacement_vector}
\end{equation}
% 
\subsubsection{Strain}
The Green-Lagrange strain tensor is given by:
%
\begin{equation}
    E_{jk} = \frac{1}{2}
    \left(
    \pdiff{u_j}{X_k} +
    \pdiff{u_k}{X_j} +
    \pdiff{u_m}{X_j} \pdiff{u_m}{X_k}
    \right),
    \qquad j, k, l = 1, 2, 3
    %
    \label{eq:green_lagrange}
\end{equation}
%
If the rotation of transverse normals are moderate, the strain-displacement relations in Eq.~\eqref{eq:green_lagrange} simplifies to von Kármán strains:
%
\begin{equation}
    \begin{Bmatrix}
        \epsilon_{xx} \\
        \epsilon_{yy} \\
        \gamma_{xy}   \\
        \gamma_{xz}   \\
        \gamma_{yz}   \\
    \end{Bmatrix} =
    {\renewcommand\arraystretch{2}
    \begin{Bmatrix}
        \dpdiff{u}{x} + \dfrac{1}{2} \left(\dpdiff{w}{x}\right)^2   \\
        \dpdiff{v}{y} + \dfrac{1}{2} \left(\dpdiff{w}{x}\right)^2   \\
        \dpdiff{u}{y} + \dpdiff{v}{x} + \dpdiff{w}{x} \dpdiff{w}{y} \\
        \dpdiff{w}{x} + \phi_x                                      \\
        \dpdiff{w}{x} + \phi_y                                      \\
    \end{Bmatrix}
    }
    + z
        {\renewcommand\arraystretch{2}
            \begin{Bmatrix}
                \dpdiff{\phi_x}{x}                      \\
                \dpdiff{\phi_y}{y}                      \\
                \dpdiff{\phi_x}{y} + \dpdiff{\phi_y}{x} \\
                0                                       \\
                0
            \end{Bmatrix}
        }
    %
    \label{eq:strain_displacement}
\end{equation}
%
or in a compact form
\begin{equation}
    \epsilon = \epsilon^0 + z \epsilon^1
    \label{eq:strain_displacement_2}
\end{equation}
%
\subsubsection{Principle of Virtual Work}
The Principle of Virtual Work (PVW) states that the sum of internal and external virtual works must be zero:
%
\begin{equation}
    \delta W^{\text{int}} + \delta W^{\text{ext}} = 0
    %
    \label{eq:pvw_1}
\end{equation}
%
The virtual internal work is related to the virtual strain energy, given by
%
\begin{equation}
    \delta W^{\text{int}} = \int_V \delta \epsilon^T \sigma ~ \text{d}V
    %
    \label{eq:strain_energy}
\end{equation}
% % >>>
The stress vector, \(\sigma\), is defined as
%
\begin{equation}
    \sigma =
    \begin{Bmatrix}
        \sigma_{xx} \\
        \sigma_{yy} \\
        \tau_{xy}   \\
        \tau_{xz}   \\
        \tau_{yz}   \\
    \end{Bmatrix}
    =
    C \left( \epsilon  - \epsilon_0 \right)
    %
    \label{eq:stress_vector}
\end{equation}
%
in which \( C \) is the material constitutive matrix and \( \epsilon_0 \) is initial deformation vector. These relations are defined as
%
\begin{equation}
    C =
    \begin{bmatrix}
        \bar{Q}_m & 0         \\
        0         & \bar{Q}_s
    \end{bmatrix}
\end{equation}
%
and
%
\begin{equation}
    \epsilon_0 =
    \begin{Bmatrix}
        \alpha \\
        0
    \end{Bmatrix} \Delta T
\end{equation}
%
Substituting Eqs.~\eqref{eq:strain_displacement_2} in Eq.~\eqref{eq:strain_energy} yields
%
\begin{equation}
    \delta W^{\text{int}} = \int_\Omega \int_{-h/2}^{h/2}
    \left(
    \delta \epsilon^T \sigma +
    z \delta \epsilon^T \sigma
    \right)
    \text{d}z ~ \text{d}\Omega
    %
    \label{eq:strain_energy_2}
\end{equation}
%
Integrating~\eqref{eq:strain_displacement_2} through the thickness, one obtains
%
\begin{equation}
    \delta W^{\text{int}} = \int_\Omega
    \left(
    \delta \epsilon_m^T N +
    \delta \epsilon_b^T M +
    \delta \epsilon_s^T Q
    \right)
    \text{d}\Omega
    %
    \label{eq:strain_energy_3}
\end{equation}
%
where the strains components, stress and thermal resultants, respectively
%
\begin{subequations}
    {\renewcommand\arraystretch{1.75}
        \begin{align}
            \label{eq:membrane_strain}
            \epsilon_m & =
            \begin{Bmatrix}
                \dpdiff{u}{x} + \dfrac{1}{2} \dpdiff{w}{x}^2                \\
                \dpdiff{v}{y} + \dfrac{1}{2} \left(\dpdiff{w}{x}\right)^2   \\
                \dpdiff{u}{y} + \dpdiff{v}{x} + \dpdiff{w}{x} \dpdiff{w}{y} \\
            \end{Bmatrix} \\
            %%%
            \label{eq:bending_strain}
            \epsilon_b & =
            \begin{Bmatrix}
                \dpdiff{\phi_x}{x}                      \\
                \dpdiff{\phi_y}{y}                      \\
                \dpdiff{\phi_x}{y} + \dpdiff{\phi_y}{x} \\
            \end{Bmatrix}                     \\
            %%%
            \label{eq:shear_strain}
            \epsilon_s & =
            \begin{Bmatrix}
                \dpdiff{w}{x} + \phi_x \\
                \dpdiff{w}{x} + \phi_y
            \end{Bmatrix}
        \end{align}
        \label{eq:strains}
    }
\end{subequations}
%
\begin{equation}
    \begin{Bmatrix}
        N \\
        M \\
        Q
    \end{Bmatrix} =
    \begin{bmatrix}
        A & B & 0   \\
        B & D & 0   \\
        0 & 0 & A_s
    \end{bmatrix}
    \begin{Bmatrix}
        \epsilon_m \\
        \epsilon_b \\
        \epsilon_s
    \end{Bmatrix}
    -
    \begin{Bmatrix}
        N_T \\
        M_T \\
        0
    \end{Bmatrix}
    = \hat{C} \hat{\epsilon} - \hat{\Psi}
    % 
    \label{eq:stress_resultants}
\end{equation}

The membrane strain, Eq.~\eqref{eq:membrane_strain}, can be considered as a linear and nonlinear parts
%
\begin{equation}
    {\renewcommand\arraystretch{1.65}
        \epsilon_m = \epsilon_m^l + \epsilon_m^{nl} =
        \begin{Bmatrix}
            \dpdiff{u}{x}                 \\
            \dpdiff{v}{y}                 \\
            \dpdiff{u}{y} + \dpdiff{v}{x} \\
        \end{Bmatrix}} +
    \begin{Bmatrix}
        \dfrac{1}{2} \left( \dpdiff{w}{x}\right)^2 \\
        \dfrac{1}{2} \left(\dpdiff{w}{x}\right)^2  \\
        \dpdiff{w}{x} \dpdiff{w}{y}                \\
    \end{Bmatrix}
\end{equation}
%
Furthermore, the strains in Eq.~\eqref{eq:strains} can written as a function of a derivative operators, that is,
%
\begin{subequations}
    \begin{align}
        \label{eq:linear_membrane_strain_operator}
        \epsilon_m^l    & =
        {\renewcommand\arraystretch{1.75}
        \begin{bmatrix}
            \dpdiff{ }{x} & 0             & 0 & 0 & 0 \\
            0             & \dpdiff{ }{y} & 0 & 0 & 0 \\
            \dpdiff{ }{y} & \dpdiff{ }{x} & 0 & 0 & 0 \\
        \end{bmatrix}
        }
        \begin{Bmatrix}
            u      \\
            v      \\
            w      \\
            \phi_x \\
            \phi_y
        \end{Bmatrix} = \mathcal{L}_m q                         \\
        %%%
        \label{eq:nonlinear_membrane_strain_operator}
        \epsilon_m^{nl} & = \frac{1}{2}
        {\renewcommand\arraystretch{1.75}
            \begin{bmatrix}
                \dpdiff{w}{x} & 0             \\
                0             & \dpdiff{w}{y} \\
                \dpdiff{w}{y} & \dpdiff{w}{x} \\
            \end{bmatrix}
            \begin{bmatrix}
                0 & 0 & \dpdiff{ }{x} & 0 & 0 \\
                0 & 0 & \dpdiff{ }{y} & 0 & 0
            \end{bmatrix}
        }
        \begin{Bmatrix}
            u      \\
            v      \\
            w      \\
            \phi_x \\
            \phi_y
        \end{Bmatrix} = \frac{1}{2} \Theta \mathcal{L}_\theta q \\
        %%%
        \label{eq:bending_strain_operator}
        \epsilon_b      & =
        {\renewcommand\arraystretch{1.75}
        \begin{bmatrix}
            0 & 0 & 0 & \dpdiff{ }{x} & 0               \\
            0 & 0 & 0 & 0             & \dpdiff{ }{y} & \\
            0 & 0 & 0 & \dpdiff{ }{y} & \dpdiff{ }{x} &
        \end{bmatrix}
        }
        \begin{Bmatrix}
            u      \\
            v      \\
            w      \\
            \phi_x \\
            \phi_y
        \end{Bmatrix} = \mathcal{L}_b q                         \\
        %%%
        \label{eq:shear_strain_operator}
        \epsilon_s      & =
        {\renewcommand\arraystretch{1.75}
        \begin{bmatrix}
            0 & 0 & \dpdiff{ }{x} & 1 & 0 \\
            0 & 0 & \dpdiff{ }{y} & 0 & 1 \\
        \end{bmatrix}
        }
        \begin{Bmatrix}
            u      \\
            v      \\
            w      \\
            \phi_x \\
            \phi_y
        \end{Bmatrix} = \mathcal{L}_s q
    \end{align}
    \label{eq:strain_operators}
\end{subequations}
%
With exception of Eq.~\eqref{eq:nonlinear_membrane_strain_operator}, that depends on the displacement, the first variation of Eq.~\eqref{eq:strain_operators} has the form:
%
\begin{equation}
    \delta \epsilon_\ast = \mathcal{L}_\ast \delta q
\end{equation}
%
while
\begin{equation}
    \delta \epsilon_m^{nl} = \Theta \mathcal{L}_\theta \delta q
\end{equation}

Now, the virtual internal work can now be written as
%
\begin{equation}
    \delta W^{\text{int}} = \int_\Omega
    \delta \hat{\epsilon}^T \hat{\sigma}
    \text{d}\Omega
    %
    \label{eq:strain_energy_4}
\end{equation}
%
in which
%
\begin{subequations}
    \begin{align}
        \delta \hat{\epsilon} & =
        \begin{bmatrix}
            \mathcal{L}_m \\
            \mathcal{L}_b \\
            \mathcal{L}_s
        \end{bmatrix} \delta q +
        \begin{bmatrix}
            \Theta \mathcal{L}_\theta \\
            0                         \\
            0
        \end{bmatrix} \delta q \\
        %%%
        \hat{\sigma}          & =
        \begin{Bmatrix}
            N \\
            M \\
            Q
        \end{Bmatrix}
    \end{align}
    \label{eq:strain_variation_sigma_hat}
\end{subequations}
% 
The virtual external work can be defined as
%
\begin{equation}
    \delta W^{\text{ext}} = -
    \left(
    \int_\Omega \delta q^T f ~ \text{d} \Omega + \delta q^T g
    \right)
\end{equation}
%
where the first term represents the external load due to loading acting on and area \( \Omega \) and second one the work done by concentrated loading, where \( f \) and \( g \) are the distributed load vector and nodal point load vector, respectively. The PVW may be now defined as
%
\begin{equation}
    \int_\Omega
    \delta \hat{\epsilon}^T \hat{\sigma}
    \text{d}\Omega =
    \int_\Omega \delta q^T f ~ \text{d} \Omega + \sum_j \delta q^j g^j
    %
    \label{eq:pvw_2}
\end{equation}

\subsubsection{Finite element discretization}
The displacement field within an element, Eq.~\eqref{eq:displacement_vector}, is given by
%
\begin{equation}
    q^\el = \hat{N} \hat{q}^\el
    %%%
    \label{eq:nodal_displacement}
\end{equation}
%
where
% 
\begin{equation}
    \hat{N} =
    \begin{bmatrix}
        \tilde{N} & \tilde{0} & \tilde{0} & \tilde{0} & \tilde{0} \\
        \tilde{0} & \tilde{N} & \tilde{0} & \tilde{0} & \tilde{0} \\
        \tilde{0} & \tilde{0} & \tilde{N} & \tilde{0} & \tilde{0} \\
        \tilde{0} & \tilde{0} & \tilde{0} & \tilde{N} & \tilde{0} \\
        \tilde{0} & \tilde{0} & \tilde{0} & \tilde{0} & \tilde{N} \\
    \end{bmatrix}
\end{equation}
%
in which \(\tilde{N} = [\tilde{N}_1, \tilde{N}_2, \cdots, \tilde{N}_n]\) is the shape functions vector, \(\tilde{0} = [0]_n\) is a null vector with \(n\) elements, in which \(n\) is the number of nodes. The local element displacement vector \(\hat{q}^\el\) is given by
% 
\begin{equation}
    \hat{q}^\el =
    \begin{Bmatrix}
        \hat{u}^\el      \\
        \hat{v}^\el      \\
        \hat{w}^\el      \\
        \hat{\phi}_x^\el \\
        \hat{\phi}_y^\el
    \end{Bmatrix}
\end{equation}
%
where, for the sake of generality, \( \hat{\gamma} = [\gamma_1, \gamma_2, \cdots, \gamma_n ]^T \) are nodal displacements. Substituting Eq.~\eqref{eq:nodal_displacement} in Eq.~\eqref{eq:pvw_2} yields
%
\begin{equation}
    {\delta \hat{q}^\el}^T
    \left(
    \int_{\Omega^\el} {B^\el}^T \hat{\sigma}^\el ~ \text{d}\Omega^\el -
    \hat{f}^\el
    \right) = 0
    %
    \label{eq:strain_energy_5}
\end{equation}
%
where \( B^\el \) is the strain-displacement matrix
%
\begin{equation}
    B^\el =
    \begin{bmatrix}
        B_m \\
        B_b \\
        B_s
    \end{bmatrix} +
    \begin{bmatrix}
        \Theta^\el B_\theta \\
        0                   \\
        0
    \end{bmatrix}
    = B^l + {B^{nl}}^\el
\end{equation}
% 
\( \hat{\sigma}^\el \) is the resultant internal forces
% 
\begin{equation}
    \hat{\sigma}^\el = \hat{C}^\el
    \left(
    B^l + \frac{1}{2} {B^{nl}}^\el
    \right) \hat{q}^\el -
    \Psi^\el
    % 
    \label{eq:sigma_hat_element}
\end{equation}
% 
and \( \hat{f}^\el \) is the equivalent nodal load vector
% 
\begin{equation}
    \hat{f}^\el = \int_{\Omega^\el} \hat{N} f^\el ~ \text{d} \Omega^\el
\end{equation}
%
With regard to Eq.~\eqref{eq:nonlinear_membrane_strain_operator}, one notices that \(\Theta^\el\) depends on the displacement within each element, and so the nonlinear part of the strain-displacement matrix.

The strain-displacement matrix \( B^\el \) is written in terms of its membrane component \( B_m \),
%
\begin{equation}
    {\renewcommand\arraystretch{2.0}
        B_m =
        \begin{bmatrix}
            \dpdiff{\tilde{N}}{x} & 0                     & 0 & 0 & 0 \\
            0                     & \dpdiff{\tilde{N}}{y} & 0 & 0 & 0 \\
            \dpdiff{\tilde{N}}{y} & \dpdiff{\tilde{N}}{x} & 0 & 0 & 0 \\
        \end{bmatrix}
    }
\end{equation}
%
bending component \(B_b\)
% 
\begin{equation}
    {\renewcommand\arraystretch{2.0}
        B_b =
        \begin{bmatrix}
            0 & 0 & 0 & \dpdiff{\tilde{N}}{x} & 0                     \\
            0 & 0 & 0 & 0                     & \dpdiff{\tilde{N}}{y} \\
            0 & 0 & 0 & \dpdiff{\tilde{N}}{y} & \dpdiff{\tilde{N}}{x}
        \end{bmatrix}
    }
\end{equation}
% 
shear component \(B_s\)
% 
\begin{equation}
    {\renewcommand\arraystretch{2.0}
        B_s =
        \begin{bmatrix}
            0 & 0 & \dpdiff{\tilde{N}}{x} & N & 0 \\
            0 & 0 & \dpdiff{\tilde{N}}{y} & 0 & N \\
        \end{bmatrix}
    }
\end{equation}
%
and gradient component \(B_\theta\)
% 
\begin{equation}
    {\renewcommand\arraystretch{2.0}
        B_\theta =
        \begin{bmatrix}
            0 & 0 & \dpdiff{\tilde{N}}{x} & 0 & 0 \\
            0 & 0 & \dpdiff{\tilde{N}}{y} & 0 & 0 \\
        \end{bmatrix}
    }
\end{equation}
%

Thus, the nonlinear equilibrium equation within the element become
\begin{equation}
    \psi^\el =
    \int_{\Omega^\el} {B^\el}^T \hat{\sigma}^\el ~ \text{d}\Omega^\el -
    \hat{f}^\el = 0
    % 
    \label{eq:residual}
\end{equation}
% 
which is also know as residual. The first term of Eq. \eqref{eq:residual} can be defined as
% 
\begin{equation}
    \int_{\Omega^\el} {B^\el}^T \hat{\sigma}^\el ~ \text{d}\Omega^\el =
    K^\el \hat{q}^\el - \hat{\Psi}^\el
\end{equation}
%
where \(K^\el\) is the standard stiffness matrix
% 
\begin{equation}
    K^\el = \int_{\Omega^\el}
    \left( B^l + {B^{nl}}^\el \right)^T \hat{C}^\el \left( B^l + \frac{1}{2} {B^{nl}}^\el \right)
    ~ \text{d}\Omega^\el
\end{equation}
% 
and \( \hat{\Psi}^\el \) is a thermal load vector
% 
\begin{equation}
    \hat{\Psi}^\el = \int_{\Omega^\el} \left( B^l + {B^{nl}}^\el \right)^T \Psi^\el
    ~ \text{d}\Omega^\el
\end{equation}


The stiffness matrix can be rewritten as a sum of linear and nonlinear stiffness matrices:
% 
\begin{equation}
    K^\el = {K^l}^\el + {K^{nl}}^\el
\end{equation}
% 
where
% 
\begin{equation}
    \begin{aligned}
        {K^l}^\el & =
        \int_{\Omega^\el} B_m^T A^\el B_m ~ \text{d}\Omega^\el +
        \int_{\Omega^\el} B_m^T B^\el B_b ~ \text{d}\Omega^\el +
        \int_{\Omega^\el} B_b^T B^\el B_m ~ \text{d}\Omega^\el \\
                  & +
        \int_{\Omega^\el} B_b^T D^\el B_b ~ \text{d}\Omega^\el +
        \int_{\Omega^\el} B_s^T A_s^\el B_s ~ \text{d}\Omega^\el
    \end{aligned}
\end{equation}
% 
and
% 
\begin{equation}
    \begin{aligned}
        {K^{nl}}^\el & =
        \frac{1}{2} \int_{\Omega^\el} B_m^T A^\el \Theta^\el B_\theta ~ \text{d}\Omega^\el +
        \frac{1}{2} \int_{\Omega^\el} B_b^T B^\el \Theta^\el B_\theta ~ \text{d}\Omega^\el \\
                     & +
        \int_{\Omega^\el} B_\theta^T {\Theta^T}^\el A^\el B_m ~ \text{d}\Omega^\el +
        \int_{\Omega^\el} B_\theta^T {\Theta^T}^\el B^\el B_b ~ \text{d}\Omega^\el         \\
                     & +
        \frac{1}{2} \int_{\Omega^\el} B_\theta^T {\Theta^T}^\el A^\el \Theta^\el B_\theta ~ \text{d}\Omega^\el
    \end{aligned}
\end{equation}
% 
In compact forms, the linear and nonlinear stiffness matrix are given by,
% 
\begin{equation}
    {K^l}^\el =
    K_{pp}^\el +
    K_{pb}^\el +
    K_{bp}^\el +
    K_{bb}^\el +
    K_{ss}^\el
\end{equation}
% 
and
% 
\begin{equation}
    {K^{nl}}^\el =
    \frac{1}{2} K_{p \theta}^\el +
    \frac{1}{2} K_{b \theta}^\el +
    K_{\theta p}^\el +
    K_{\theta b}^\el +
    \frac{1}{2} K_{\theta \theta}^\el
\end{equation}
% 
respectively. The thermal load vector can be written as
% 
\begin{equation}
    \hat{\Psi}^\el =
    \int_{\Omega^\el} B_m^T N_T^\el ~ \text{d}\Omega^\el +
    \int_{\Omega^\el} B_b^T M_T^\el ~ \text{d}\Omega^\el +
    \int_{\Omega^\el} B_\theta^T {\Theta^T}^\el  N_T^\el ~ \text{d}\Omega^\el
    % 
    \label{eq:element_thermal_load}
\end{equation}
% 
The last term of Eq.~\eqref{eq:element_thermal_load} may be rewritten as
% 
\begin{equation}
    \int_{\Omega^\el} B_\theta^T {\Theta^T}^\el  N_T^\el ~ \text{d}\Omega^\el
    =
    \left(
    \int_{\Omega^\el}
    B_\theta^T
    {\renewcommand\arraystretch{1.15}
        \begin{bmatrix}
            N_T^1 & N_T^3 \\
            N_T^3 & N_T^2
        \end{bmatrix}
    }
    B_\theta ~ \text{d}\Omega^\el
    \right) \hat{q}^\el
    = K_T^\el
\end{equation}
% 
Thus, one may define the thermal load vector as
% 
\begin{equation}
    \hat{\Psi}^\el = \hat{f}^\el_{{N}_T} + \hat{f}^\el_{{M}_T} + K_T^\el
\end{equation}

The residual can now be rewritten as
% 
\begin{equation}
    \psi^\el = \bar{K}^\el \hat{q}^\el - \bar{f}^\el
    % 
    \label{eq:residual_2}
\end{equation}
% 
where \(\bar{K}^\el\) is given by
% 
\begin{equation}
    \bar{K}^\el = {K^l}^\el + {K^{nl}}^\el - K_T^\el
\end{equation}
% 
and
% 
\begin{equation}
    \bar{f}^\el =
    \hat{f}^\el +
    \hat{f}^\el_{{N}_T} +
    \hat{f}^\el_{{M}_T}
\end{equation}

\subsubsection{Solution to nonlinear equilibrium equations}
An approximation of the residual (unbalanced forces) may be obtained by equating to zero the linearized Taylor's series expansion of \( \psi^\el\) in the neighborhood of an initial estimate of the total displacement, \( \hat{q}^\el_i \), as
%
\begin{equation}
    \psi^\el ( \hat{q}^\el_{i+1} ) \simeq
    \psi^\el ( \hat{q}^\el_i ) +
    T^\el \Delta \hat{q}^\el_{i} = 0
\end{equation}
%
where \( T^\el \) is known as the tangent stiffness matrix. It is evaluated as
%
\begin{equation}
    T^\el = \pdiff{\psi^\el ( \hat{q}^\el_i )}{\hat{q}^\el_i}
\end{equation}
%
The tangent matrix may be written as
% 
\begin{equation}
    {\renewcommand\arraystretch{2.5}
        T^\el =
        \begin{bmatrix}
            \dpdiff{\psi^\el_{1}}{\hat{q}^\el_{1}} & \dpdiff{\psi^\el_{1}}{\hat{q}^\el_{2}} & \cdots & \dpdiff{\psi^\el_{1}}{\hat{q}^\el_{m}} \\
            \dpdiff{\psi^\el_{2}}{\hat{q}^\el_{1}} & \dpdiff{\psi^\el_{2}}{\hat{q}^\el_{2}} & \cdots & \dpdiff{\psi^\el_{2}}{\hat{q}^\el_{m}} \\
            \vdots                                 & \vdots                                 & \ddots & \vdots                                 \\
            \dpdiff{\psi^\el_{m}}{\hat{q}^\el_{1}} & \dpdiff{\psi^\el_{m}}{\hat{q}^\el_{2}} & \cdots & \dpdiff{\psi^\el_{m}}{\hat{q}^\el_{m}} \\
        \end{bmatrix}
    }
\end{equation}
%
where \(m\) is the element number of degrees of freedom. For convenience, the tangent matrix is rewritten with respect to Eq.~\eqref{eq:residual} as
%
\begin{equation}
    T^\el = \int_{\Omega^\el}
    \left(
    B^\el \pdiff{\hat{\sigma}^\el}{\hat{q}^\el} +
    \pdiff{B^\el}{\hat{q}^\el} \hat{\sigma}^\el
    \right)
    ~ \text{d}\Omega^\el
    % 
    \label{eq:tangent_matrix}
\end{equation}
%
The first term of Eq.~\eqref{eq:tangent_matrix} may be rewritten as (see Eq.~\eqref{eq:sigma_hat_element})
% 
\begin{equation}
    \begin{aligned}
        \int_{\Omega^\el} {B^\el}^T \pdiff{\hat{\sigma}^\el}{\hat{q}^\el} ~ \text{d}\Omega^\el
         & =
        \int_{\Omega^\el} {B^\el}^T \hat{C}^\el \pdiff{}{\hat{q}^\el}
        \left[
            \left( B^l + \frac{1}{2}{B^{nl}}^\el \right) \hat{q}^\el -
            \Psi^\el
        \right] ~ \text{d} \Omega^\el                                      \\
         & =
        \int_{\Omega^\el} {B^\el}^T \hat{C}^\el B^\el ~ \text{d}\Omega^\el \\
         & =
        {K^l}^\el + {K^{nl_2}}^\el
    \end{aligned}
    %
    \label{eq:first_term_tangent_matrix}
\end{equation}
% 
where 
% 
\begin{equation}
    {K^{nl_2}}^\el = 
    K_{p \theta}^\el +
    K_{b \theta}^\el +
    K_{\theta p}^\el +
    K_{\theta b}^\el +
    K_{\theta \theta}^\el
\end{equation}
%
The second term of Eq.~\eqref{eq:tangent_matrix} is written as
%
\begin{equation}
    \begin{aligned}
        \int_{\Omega^\el} \pdiff{{B^\el}^T}{\hat{q}^\el} \hat{\sigma}^\el ~ \text{d}\Omega^\el
         & =
        \int_{\Omega^\el} B_\theta^T \pdiff{{\Theta^\el}^T}{\hat{q}^\el} N^\el ~ \text{d}\Omega^\el \\
         & =
        \int_{\Omega^\el} B_\theta^T
        {\renewcommand\arraystretch{2.0}
        \begin{bmatrix}
            \dpdiff{}{\hat{q}^\el} \left( \dpdiff{w^\el}{x} \right) N^\el_1 +
            \dpdiff{}{\hat{q}^\el} \left( \dpdiff{w^\el}{y} \right) N^\el_3 \\
            % 
            \dpdiff{}{\hat{q}^\el} \left( \dpdiff{w^\el}{x} \right) N^\el_3 +
            \dpdiff{}{\hat{q}^\el} \left( \dpdiff{w^\el}{y} \right) N^\el_2 
        \end{bmatrix}} ~ \text{d}\Omega^\el            \\
         & =
        \int_{\Omega^\el} B_\theta^T
        {\renewcommand\arraystretch{1.15}
        \begin{bmatrix}
            N^\el_1 & N^\el_3 \\
            N^\el_3 & N^\el_2
        \end{bmatrix}
        \begin{Bmatrix}
            \dpdiff{}{\hat{q}^\el} \left( \dpdiff{w^\el}{x} \right) \\
            \dpdiff{}{\hat{q}^\el} \left( \dpdiff{w^\el}{y} \right)
        \end{Bmatrix}}
        ~ \text{d}\Omega^\el
    \end{aligned}
    %
    \label{eq:second_term_tangent_matrix}
\end{equation}
% 
The derivative terms may be evaluated as
% 
\begin{equation}
    \dpdiff{}{\hat{q}^\el_j}
    \left[
        \dpdiff{\left(\tilde{N}^\alpha \hat{w}^\el_\alpha\right)}{\lambda}
        \right] =
    \begin{cases}
        \dpdiff{\tilde{N}}{\lambda}, & \text{if} ~ \hat{q}^\el_j = \hat{w}^\el_\alpha \\[10pt]
        0,                           & \text{otherwise}
    \end{cases}
\end{equation}
% 
which is the definition of the strain-displacement matrix \(B_\theta\). Thus,
\begin{equation}
    \begin{aligned}
        \int_{\Omega^\el} \pdiff{{B^\el}^T}{\hat{q}^\el} \hat{\sigma}^\el
         & =
        \int_{\Omega^\el}
        B_\theta^T
        {\renewcommand\arraystretch{1.15}
        \begin{bmatrix}
            N^\el_{1} & N^\el_{3} \\
            N^\el_{3} & N^\el_{2}
        \end{bmatrix}}
        B_\theta ~ \text{d}\Omega^\el \\
         & = {K^\sigma}^\el
    \end{aligned}
\end{equation}
% 
Finally, the tangent stiffness matrix is given by
% 
\begin{equation}
    T^\el = {K^l}^\el + {K^{nl_2}}^\el + {K^\sigma}^\el
\end{equation}
\end{document}